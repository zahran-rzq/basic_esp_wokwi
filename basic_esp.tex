\documentclass[12pt,a4paper]{article}
\usepackage[utf8]{inputenc}
\usepackage[bahasa]{babel}
\usepackage{amsmath}
\usepackage{amsfonts}
\usepackage{amssymb}
\usepackage{graphicx}
\usepackage{xcolor}
\usepackage{listings}
\usepackage{geometry}
\usepackage{hyperref}
\usepackage{tcolorbox}
\usepackage{enumitem}

\geometry{left=2.5cm,right=2.5cm,top=2.5cm,bottom=2.5cm}

% Setup untuk kode
\definecolor{codegreen}{rgb}{0,0.6,0}
\definecolor{codegray}{rgb}{0.5,0.5,0.5}
\definecolor{codepurple}{rgb}{0.58,0,0.82}
\definecolor{backcolour}{rgb}{0.95,0.95,0.92}

\lstdefinestyle{mystyle}{
    backgroundcolor=\color{backcolour},   
    commentstyle=\color{codegreen},
    keywordstyle=\color{magenta},
    numberstyle=\tiny\color{codegray},
    stringstyle=\color{codepurple},
    basicstyle=\ttfamily\footnotesize,
    breakatwhitespace=false,         
    breaklines=true,                 
    captionpos=b,                    
    keepspaces=true,                 
    numbers=left,                    
    numbersep=5pt,                  
    showspaces=false,                
    showstringspaces=false,
    showtabs=false,                  
    tabsize=2
}

\lstset{style=mystyle}

\title{\textbf{Materi Dasar Pemrograman ESP32}}
\author{Pelatihan IEEE 2025}
\date{\today}

\begin{document}

\maketitle
\tableofcontents
\newpage

\section{Dasar-Dasar Pemrograman ESP32}

\subsection{Contoh Program 1: Blink LED Internal}

\subsubsection{Kode Program}
\begin{lstlisting}[language=C++, caption={Program Blink LED}]
void setup() {
  pinMode(2, OUTPUT);  // GPIO2 = LED internal di board ESP32
}

void loop() {
  digitalWrite(2, HIGH);  // Nyalakan LED
  delay(1000);            // Tunggu 1 detik
  digitalWrite(2, LOW);   // Matikan LED
  delay(1000);            // Tunggu 1 detik
}
\end{lstlisting}

\subsubsection{Penjelasan Kode}
\begin{itemize}
    \item \texttt{void setup()}: Fungsi yang dijalankan sekali saat ESP32 pertama kali dinyalakan atau di-reset
    \item \texttt{pinMode(2, OUTPUT)}: Mengonfigurasi GPIO pin 2 sebagai output. Parameter pertama adalah nomor pin, parameter kedua adalah mode (OUTPUT/INPUT)
    \item \texttt{void loop()}: Fungsi yang dijalankan berulang-ulang setelah \texttt{setup()} selesai
    \item \texttt{digitalWrite(2, HIGH)}: Memberikan tegangan HIGH (3.3V) pada pin 2, membuat LED menyala
    \item \texttt{delay(1000)}: Menghentikan eksekusi program selama 1000 milidetik (1 detik)
    \item \texttt{digitalWrite(2, LOW)}: Memberikan tegangan LOW (0V) pada pin 2, membuat LED mati
\end{itemize}

\begin{tcolorbox}[colback=blue!5!white,colframe=blue!75!black,title=Hasil]
LED internal berkedip setiap 1 detik (nyala 1 detik, mati 1 detik).
\end{tcolorbox}

\newpage
\section{Pertemuan 2: Input \& Output Digital}

\subsection{Tujuan}
\begin{itemize}
    \item Menggunakan tombol (push button) sebagai input
    \item Mengendalikan LED dengan tombol
    \item Memahami konsep debouncing sederhana
\end{itemize}

\subsection{Komponen di Wokwi}
\begin{itemize}
    \item ESP32
    \item 1 Push Button
    \item 1 LED
    \item 1 Resistor (220 $\Omega$)
\end{itemize}

\textbf{Catatan:} Aktifkan internal pull-up agar tidak perlu resistor tambahan pada tombol.

\subsection{Contoh Program 2: LED ON saat Tombol Ditekan}

\subsubsection{Kode Program}
\begin{lstlisting}[language=C++, caption={Kontrol LED dengan Tombol}]
const int ledPin = 2;
const int buttonPin = 4;

void setup() {
  pinMode(ledPin, OUTPUT);
  pinMode(buttonPin, INPUT_PULLUP); // tombol aktif LOW
}

void loop() {
  int tombol = digitalRead(buttonPin); // baca status tombol
  if (tombol == LOW) { // ditekan
    digitalWrite(ledPin, HIGH);
  } else {
    digitalWrite(ledPin, LOW);
  }
}
\end{lstlisting}

\subsubsection{Penjelasan Kode}
\begin{itemize}
    \item \texttt{const int ledPin = 2}: Mendefinisikan konstanta untuk pin LED (tidak dapat diubah)
    \item \texttt{const int buttonPin = 4}: Mendefinisikan konstanta untuk pin tombol
    \item \texttt{pinMode(buttonPin, INPUT\_PULLUP)}: Mengaktifkan resistor pull-up internal ESP32 (sekitar 45k$\Omega$). Ini membuat pin dalam kondisi HIGH secara default
    \item \texttt{digitalRead(buttonPin)}: Membaca status digital pin tombol (HIGH atau LOW)
    \item \texttt{if (tombol == LOW)}: Karena menggunakan INPUT\_PULLUP, tombol yang ditekan akan memberikan nilai LOW (ground)
    \item Logika: Jika tombol ditekan (LOW), LED menyala. Jika tidak ditekan (HIGH), LED mati
\end{itemize}

\newpage
\section{Analog Input (Sensor Potensiometer)}

\subsection{Tujuan}
\begin{itemize}
    \item Membaca nilai analog menggunakan \texttt{analogRead()}
    \item Menampilkan hasil ke Serial Monitor
\end{itemize}

\subsection{Komponen}
\begin{itemize}
    \item ESP32
    \item Potensiometer (hubungkan ke pin 34 atau 35)
    \item LED (opsional)
\end{itemize}

\subsection{Kode: Membaca Potensiometer}
\begin{lstlisting}[language=C++, caption={Membaca Nilai Analog}]
const int potPin = 34;

void setup() {
  Serial.begin(115200); // buka komunikasi serial
}

void loop() {
  int nilai = analogRead(potPin);
  Serial.println(nilai);
  delay(200);
}
\end{lstlisting}

\subsection{Penjelasan Kode}
\begin{itemize}
    \item \texttt{Serial.begin(115200)}: Menginisialisasi komunikasi serial dengan baud rate 115200 bps (bit per second)
    \item \texttt{analogRead(potPin)}: Membaca nilai analog dari pin 34. ESP32 memiliki ADC (Analog to Digital Converter) 12-bit
    \item \textbf{Rentang nilai}: 0 - 4095 ($2^{12} = 4096$ level)
    \begin{itemize}
        \item 0 = 0V
        \item 4095 = 3.3V
    \end{itemize}
    \item \texttt{Serial.println(nilai)}: Mengirim nilai ke Serial Monitor dengan baris baru
    \item \texttt{delay(200)}: Menunda 200ms agar output tidak terlalu cepat
\end{itemize}

\begin{tcolorbox}[colback=green!5!white,colframe=green!75!black,title=Catatan]
Nilai ADC ESP32: \textbf{0--4095} (0--3.3V). Dapat digunakan untuk mengatur kecepatan motor, kecerahan LED, dll.
\end{tcolorbox}

\newpage
\section{PWM (Pulse Width Modulation)}

\subsection{Mengatur Kecerahan LED}

\subsubsection{Kode Program}
\begin{lstlisting}[language=C++, caption={Kontrol Kecerahan LED dengan PWM}]
int ledPin = 2; //deklarasi pin led1

void setup() {
  pinMode(ledPin, OUTPUT); //pin led dijadikan sebagai pin output
  Serial.begin(9600); //memulai komunikasi ke serial monitor
}

void loop() {
  analogWrite(ledPin, 0); //led mati
  delay(2000);
  analogWrite(ledPin, 50); //led nyala redup
  delay(2000);
  analogWrite(ledPin, 255); //led nyala terang
  delay(2000);
  analogWrite(ledPin, 50); //led nyala redup
  delay(2000); //PWM dari mati, nyala redup, terang, redup
}
\end{lstlisting}

\subsubsection{Penjelasan Kode}
\begin{itemize}
    \item \textbf{PWM (Pulse Width Modulation)}: Teknik untuk mengontrol daya dengan mengubah duty cycle sinyal digital
    \item \texttt{analogWrite(pin, value)}: Menghasilkan sinyal PWM pada pin
    \begin{itemize}
        \item \textbf{value = 0}: LED mati (0\% duty cycle)
        \item \textbf{value = 50}: LED redup ($\approx$20\% duty cycle)
        \item \textbf{value = 255}: LED terang penuh (100\% duty cycle)
    \end{itemize}
    \item Nilai PWM: \textbf{0-255} (8-bit)
    \item ESP32 memiliki 16 channel PWM dengan resolusi hingga 16-bit
    \item Duty cycle = $\frac{\text{value}}{255} \times 100\%$
\end{itemize}

\newpage
\section{Kontrol Servo dengan Potensiometer}

\subsection{Kode Program}
\begin{lstlisting}[language=C++, caption={Kontrol Servo Berdasarkan Potensiometer}]
#include <ESP32Servo.h>

Servo myservo;
const int potPin = 34;

void setup() {
  myservo.attach(15); // pin servo di GPIO15
}

void loop() {
  int potValue = analogRead(potPin);
  int angle = map(potValue, 0, 4095, 0, 180);
  myservo.write(angle);
  delay(20);
}
\end{lstlisting}

\subsection{Penjelasan Kode}
\begin{itemize}
    \item \texttt{\#include <ESP32Servo.h>}: Mengimpor library untuk mengontrol servo motor
    \item \texttt{Servo myservo}: Membuat objek servo dengan nama myservo
    \item \texttt{myservo.attach(15)}: Menghubungkan objek servo ke pin GPIO 15
    \item \texttt{map(potValue, 0, 4095, 0, 180)}: Fungsi mapping untuk mengkonversi nilai
    \begin{itemize}
        \item Input: 0-4095 (nilai ADC)
        \item Output: 0-180 (sudut servo dalam derajat)
        \item Formula: \\
        $\text{output} = \frac{(\text{input} - \text{inMin}) \times (\text{outMax} - \text{outMin})}{\text{inMax} - \text{inMin}} + \text{outMin}$
    \end{itemize}
    \item \texttt{myservo.write(angle)}: Menggerakkan servo ke sudut tertentu
    \item \texttt{delay(20)}: Delay singkat untuk stabilisasi servo
\end{itemize}

\begin{tcolorbox}[colback=yellow!5!white,colframe=yellow!75!black,title=Catatan]
Servo di Wokwi membutuhkan supply 5V dan sinyal pada pin yang mendukung PWM.
\end{tcolorbox}

\newpage
\section{Komunikasi Serial}

\subsection{Tujuan}
\begin{itemize}
    \item Memahami cara berkomunikasi dengan PC via Serial Monitor
    \item Dapat menerima perintah teks dan mengontrol perangkat (LED/servo)
\end{itemize}

\subsection{Contoh Kode: Perintah ON/OFF LED}

\begin{lstlisting}[language=C++, caption={Kontrol LED via Serial}]
const int ledPin = 2;
String input;

void setup() {
  Serial.begin(115200);
  pinMode(ledPin, OUTPUT);
  Serial.println("Ketik ON atau OFF:");
}

void loop() {
  if (Serial.available()) {
    input = Serial.readStringUntil('\n');
    input.trim();

    if (input == "ON") {
      digitalWrite(ledPin, HIGH);
      Serial.println("LED ON");
    } else if (input == "OFF") {
      digitalWrite(ledPin, LOW);
      Serial.println("LED OFF");
    } else {
      Serial.println("Perintah tidak dikenal");
    }
  }
}
\end{lstlisting}

\subsection{Penjelasan Kode}
\begin{itemize}
    \item \texttt{String input}: Variabel untuk menyimpan string input dari serial
    \item \texttt{Serial.available()}: Mengecek apakah ada data yang tersedia di buffer serial (return true jika ada)
    \item \texttt{Serial.readStringUntil('\textbackslash n')}: Membaca karakter sampai menemukan newline
    \item \texttt{input.trim()}: Menghapus whitespace (spasi, tab, newline) di awal dan akhir string
    \item \texttt{if (input == "ON")}: Membandingkan string input dengan "ON"
    \item \textbf{Flow}:
    \begin{enumerate}
        \item Cek apakah ada data serial
        \item Baca string sampai enter
        \item Bersihkan spasi
        \item Eksekusi perintah sesuai input
    \end{enumerate}
\end{itemize}

\newpage
\section{PROJECT AKHIR: Kontrol Servo via Perintah Serial}

\subsection{Tujuan}
Membuat sistem yang dapat menerima perintah dari Serial Monitor untuk mengontrol posisi servo (simulasi motor DC).

\subsection{Komponen Wokwi}
\begin{itemize}
    \item ESP32 DevKit v1
    \item Servo Motor
    \item LED indikator (opsional)
\end{itemize}

\textbf{Hubungan pin:}
\begin{itemize}
    \item Servo $\rightarrow$ pin 15
    \item LED $\rightarrow$ pin 2 (opsional)
\end{itemize}

\subsection{Kode Lengkap: Kontrol Servo via Serial}

\begin{lstlisting}[language=C++, caption={Project: Kontrol Servo dengan Serial}]
#include <Servo.h>

Servo motor;
String command;  // perintah dari serial
int angle = 90;  // posisi awal

void setup() {
  Serial.begin(115200);
  motor.attach(15);
  motor.write(angle);
  Serial.println("Ketik perintah: LEFT, RIGHT, CENTER, atau ANGLE <nilai>");
}

void loop() {
  if (Serial.available()) {
    command = Serial.readStringUntil('\n');
    command.trim();

    if (command == "LEFT") {
      angle = 0;
    } else if (command == "RIGHT") {
      angle = 180;
    } else if (command == "CENTER") {
      angle = 90;
    } else if (command.startsWith("ANGLE")) {
      int value = command.substring(6).toInt();
      if (value >= 0 && value <= 180) angle = value;
    } else {
      Serial.println("Perintah tidak dikenal");
      return;
    }

    motor.write(angle);
    Serial.print("Servo di posisi: ");
    Serial.println(angle);
  }
}
\end{lstlisting}

\subsection{Penjelasan Kode}
\begin{itemize}
    \item \texttt{motor.write(angle)}: Menggerakkan servo ke posisi awal (90$^\circ$)
    \item \texttt{command.startsWith("ANGLE")}: Mengecek apakah string dimulai dengan "ANGLE"
    \item \texttt{command.substring(6)}: Mengambil substring mulai dari karakter ke-6 (setelah "ANGLE ")
    \begin{itemize}
        \item Contoh: "ANGLE 45" $\rightarrow$ substring(6) = "45"
    \end{itemize}
    \item \texttt{.toInt()}: Mengkonversi string menjadi integer
    \item \textbf{Validasi}: \texttt{if (value >= 0 \&\& value <= 180)} memastikan nilai sudut valid
    \item \texttt{return}: Keluar dari fungsi loop() jika perintah tidak dikenal
\end{itemize}

\textbf{Perintah yang didukung:}
\begin{itemize}
    \item \texttt{LEFT} $\rightarrow$ Servo ke 0$^\circ$
    \item \texttt{RIGHT} $\rightarrow$ Servo ke 180$^\circ$
    \item \texttt{CENTER} $\rightarrow$ Servo ke 90$^\circ$
    \item \texttt{ANGLE 45} $\rightarrow$ Servo ke 45$^\circ$ (atau nilai 0-180)
\end{itemize}

\newpage
\section{BONUS: DHT22 + OLED Display}

\subsection{Kode Program}
\begin{lstlisting}[language=C++, caption={Sensor DHT22 dengan OLED Display}]
#include <Wire.h>
#include <Adafruit_SSD1306.h>
#include "DHT.h"

#define SCREEN_WIDTH 128
#define SCREEN_HEIGHT 64
Adafruit_SSD1306 display(SCREEN_WIDTH, SCREEN_HEIGHT, &Wire, -1);

#define DHTPIN 15
#define DHTTYPE DHT22
DHT dht(DHTPIN, DHTTYPE);

void setup() {
  Serial.begin(115200);
  dht.begin();
  display.begin(SSD1306_SWITCHCAPVCC, 0x3C);
  display.clearDisplay();
  display.setTextSize(1);
  display.setTextColor(SSD1306_WHITE);
}

void loop() {
  float h = dht.readHumidity();
  float t = dht.readTemperature();

  display.clearDisplay();
  display.setCursor(0, 0);
  display.println("Sensor DHT22");
  display.print("Suhu: ");
  display.print(t);
  display.println(" C");
  display.print("Lembab: ");
  display.print(h);
  display.println(" %");
  display.display();

  delay(2000);
}
\end{lstlisting}

\subsection{Penjelasan Kode}
\begin{itemize}
    \item \texttt{\#include <Wire.h>}: Library untuk komunikasi I$^2$C
    \item \texttt{\#include <Adafruit\_SSD1306.h>}: Library untuk OLED display SSD1306
    \item \texttt{\#include "DHT.h"}: Library untuk sensor DHT22
    \item \texttt{\#define SCREEN\_WIDTH 128}: Mendefinisikan lebar layar OLED (128 pixel)
    \item \texttt{\#define SCREEN\_HEIGHT 64}: Mendefinisikan tinggi layar OLED (64 pixel)
    \item \texttt{Adafruit\_SSD1306 display(...)}: Membuat objek display
    \begin{itemize}
        \item \texttt{\&Wire}: Pointer ke objek I$^2$C
        \item \texttt{-1}: Reset pin (tidak digunakan)
    \end{itemize}
    \item \texttt{\#define DHTTYPE DHT22}: Menentukan tipe sensor (DHT22/DHT11)
    \item \texttt{DHT dht(DHTPIN, DHTTYPE)}: Membuat objek sensor DHT
    \item \texttt{display.begin(SSD1306\_SWITCHCAPVCC, 0x3C)}: 
    \begin{itemize}
        \item \texttt{SSD1306\_SWITCHCAPVCC}: Mode power internal
        \item \texttt{0x3C}: Alamat I$^2$C OLED (biasanya 0x3C atau 0x3D)
    \end{itemize}
    \item \texttt{dht.readHumidity()}: Membaca kelembaban (float, satuan \%)
    \item \texttt{dht.readTemperature()}: Membaca suhu (float, satuan $^\circ$C)
    \item \texttt{display.clearDisplay()}: Membersihkan buffer display
    \item \texttt{display.setCursor(0, 0)}: Set posisi kursor (x=0, y=0)
    \item \texttt{display.println()}: Menampilkan teks dengan pindah baris
    \item \texttt{display.display()}: Menampilkan buffer ke layar OLED (wajib dipanggil!)
\end{itemize}

\begin{tcolorbox}[colback=red!5!white,colframe=red!75!black,title=Catatan Penting]
Wokwi otomatis mengenali alamat I$^2$C OLED (0x3C).
\end{tcolorbox}

\newpage
\section{Rangkuman Konsep Penting}

\subsection{Fungsi Dasar Arduino/ESP32}
\begin{itemize}
    \item \texttt{pinMode()} - Set mode pin
    \item \texttt{digitalWrite()} - Tulis digital
    \item \texttt{digitalRead()} - Baca digital
    \item \texttt{analogWrite()} - Tulis PWM
    \item \texttt{analogRead()} - Baca analog
    \item \texttt{delay()} - Tunda waktu
\end{itemize}

\subsection{Komunikasi Serial}
\begin{itemize}
    \item \texttt{Serial.begin()} - Inisialisasi
    \item \texttt{Serial.print()} - Kirim data
    \item \texttt{Serial.available()} - Cek data tersedia
    \item \texttt{Serial.readStringUntil()} - Baca string
\end{itemize}

\subsection{Tipe Data}
\begin{itemize}
    \item \texttt{int} - Integer (-32768 to 32767)
    \item \texttt{float} - Bilangan desimal
    \item \texttt{String} - Teks
    \item \texttt{const} - Konstanta
\end{itemize}

\subsection{Struktur Kontrol}
\begin{itemize}
    \item \texttt{if-else} - Percabangan
    \item \texttt{loop()} - Perulangan otomatis
    \item \texttt{for/while} - Perulangan manual
\end{itemize}

\vspace{1cm}
\begin{center}
\line(1,0){400}\\
\textbf{Dibuat untuk pembelajaran ESP32 di Wokwi Simulator}\\
\textbf{Pelatihan IEEE 2025}
\end{center}

\end{document}
